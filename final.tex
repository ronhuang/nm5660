% final.tex
% Final Report for NM5660 Independent Study Module
% Author: Yih-Lun Huang
% Revisions: 16 April 2010

\documentclass[a4paper,titlepage]{article}

\usepackage{verbatim}
\usepackage{natbib}
\usepackage{url}
\usepackage{tikz}

\hyphenation{trans-pa-ren-cy}


\begin{document}


% TODO: change to ``Toward...''?
\title{Investigation of the relationship between transparency and user
  experience}
\author{Yih-Lun Huang\\
HY090191Y\\
\texttt{ylhuang@nus.edu.sg}}
\maketitle


\begin{abstract}
[FIXME]
\end{abstract}


\tableofcontents
\newpage


\section{Introduction}
\label{sec:introduction}
The notion of user experience has been wildly adopted in the field of
human-computer interaction. However, as many existing literature has
already pointed out \citep{ux:hassenzahl, ux:law}, a common agreement
on the exact definition of user experience is still in debate, and so
far is not fully understood by researchers, consultants, managers in
the industry, and practitioners in the field.

They do agree that the traditional usability framework, which focus
primarily on work-related and performance-based use, is limited and
can no longer model the ever more complex settings of modern
computation. This was especially true when ubiquitous computing was
introduced, where the interfaces between users and computers are no
longer single, static, and restricted to the traditional window, icon,
menu, and pointing device. Instead, multiple interfaces are spread
throughout the environment and are updated dynamically depending on
the context \citep{windows:bolter}.

But besides the above agreement that the traditional usability
framework is not enough, people from different backgrounds and
interests, with varying focuses and concerns, have formulated their
own definitions and statements on user experience. Some tackle this
from the emotional and affective aspects of human-computer
interaction, such as stressing the importance of emotions as
consequences of product use \citep{emotions:desmet}. Some deal with
the nature of experience or the experiential aspect,
e.g.\ \citet{experience:forlizzi}. And yet others address the human
needs beyond the ``functional'' \citep{emotional:norman}, such as
surprise or intimacy. \citep{alternatives:gaver}.

Though very exciting to witness and take part in this particular time
frame, where opportunities are still around to make significant
contributions to the human knowledge, it is actually quite troublesome
for practitioners in the field of human-computer interaction. And in
the case of this article, we specifically focus our attention on
software developers.

Software developers are the focus group for this article. More often
than not, researchers in the user experience field concentrate their
attentions primarily on two set of people: designers and
end-users. But the fact is that, software developers in small and
medium sized enterprises, which have a dominating percentage for most
economies\footnote{\url{http://epp.eurostat.ec.europa.eu/portal/page/portal/european_business/special_topics/small_medium_sized_enterprises_SMEs}},
usually take on a number of roles, such as being designers, user
experience engineers and analysts. Hence in order to effectively
improve the user experience of human-computer interaction, we believe
that a more practical measure adapted toward software developers are
required.

In this article, we argue that patterns of user experience---a
collection of the best practices within the user experience design
domain, transcribed in languages similar to software patterns---is a
good tool for helping the software developers on improving the user
experience for end-users. We also propose experiments on collecting
patterns of user experience, with particular interest in balancing the
transparency of user interface.

In the following sections, we first go through the related works of
our study (section \ref{sec:related}). Afterward, we cover the
backgrounds on patterns, user experience, and transparency (section
\ref{sec:backgrounds}). Next, we discuss about patterns of user
experience (section \ref{sec:pux}) and propose experiments to collect
them (section \ref{sec:experiment}). Finally, the conclusion and
future works (section \ref{sec:conclusion}).


\section{Related Works}
\label{sec:related}
\citet{language:alexander} and its companion book
\citet{timeless:alexander} established the concept of patterns. The
publications featured a very complete set of patterns on improving the
quality of life for people who are living inside physical buildings or
in city. Although the patterns target the architecture and urban
planning, they can also apply to many disciplines, including software
development.

Patterns are no strangers to software developers. The publication of
\textit{Design Patterns} \citep{patterns:gamma} dramatically changed
the practice of software development. Since then, many authors have
written books on software patterns covering various aspects of
software engineering, This includes: programming paradigms,
programming languages, software development methodologies,
etc. However, these patterns are designed to help the software
developers, instead of helping the actual users who is using the
software.

There are also quite a few publications discussing the patterns of
user experience. \citet{participatory:dearden} examined the patterns
for enabling users to participate in the product design.
\citet{speech:zajicek} suggested patterns of good speech interface
design for older adults. \citet{unix:raymond} provided patterns to
improve the user experience of command-line interface.
\citet{patterns:tidwell} described patterns for graphical user
interfaces. But none of the publications explicitly examined
transparency within the patterns of user experience.


\section{Backgrounds}
\label{sec:backgrounds}
Before we discuss the patterns of user experience and transparency, we
would like to break them down and clarify the backgrounds and view
points from where we are perceiving from.

\subsection{Patterns}
Software developers, like engineers in the other engineering
disciplines, rely on standards---codified and enforced from good
practices---to improve the quality and lower the cost of software
development \citep{practice:ipenz}. These standards can be in the form
of examples, guidelines or even abstractified into patterns.

Patterns, as described by \citet{patterns:griffiths}, are abstractions
from the specific examples, which give patterns their generative
power. They do not supply immediate solutions to the problems,
however. Creativity is required for the software developers to
implement the patterns. Patterns are also considered more advantageous
than guidelines because they are more related to a context and are
problem centered \citep{patterns:welie}.

Software patterns are of great values to software developers. They
provided valuable engineering solutions to difficult technical
problems or elegant tricks to build user interfaces. Mastering
software patterns is so influential that it is often regarded as an
indication to distinguishing rockstar software developers from
mediocre ones \citep{rockstar:iskold}. But what are patterns?

The term ``pattern'' originated from \citet{timeless:alexander}, where
the author was looking for structures of urban planning and building
architecture that have positive influence on people by improving their
comfort and quality of life. He described:
\begin{quote}
  {\it ``Each pattern is a three-part rule, which expresses a relation
    between a certain context, a problem, and a solution.

    As an element in the world, each pattern is a relationship between
    a certain context, a certain system of forces which occurs
    repeatedly in that context, and a certain spatial configuration
    which allows these forces to resolve themselves.

    As an element of language, a pattern is an instruction, which
    shows how this spatial configuration can be used, over and over
    again, to resolve the given system of forces, wherever the context
    makes it relevant.

    The pattern is, in short, at the same time a thing, which happens
    in the world, and the rule which tells us how to create that
    thing, and when we must create it. It is both a process and a
    thing; both a description of a thing which is alive, and a
    description of the process which will generate that thing.''}

  \raggedleft \citep[p. 247]{timeless:alexander}
\end{quote}

Several years later, this idea of patterns was introduced to the
software development domain by \citet{patterns:beck} for designing
user interfaces with Smalltalk. This eventually led to one of the most
important publications in software development, \textit{Design
  Patterns: Elements of Reusable Object-Oriented Software} from
\citet{patterns:gamma}.

Though many have given their own definitions of patterns from the
software perspective, most of them still closely resemble to the
original definition from Alexander. For example:
\begin{quote}
  {\it ``A pattern is a named nugget of instructive information that
    captures the essential structure and insight of a successful
    family of proven solutions to a recurring problem that arises
    within a certain context and system of forces.''}

  \raggedleft \citet{patterns:appleton}
\end{quote}

\begin{quote}
  {\it ``Each pattern is a three-part rule, which expresses a relation
    between a certain context, a certain system of forces which occurs
    repeatedly in that context, and a certain software configuration
    which allows these forces to resolve themselves.''}

  \raggedleft \citet{timeless:gabriel}
\end{quote}

\subsection{Clarifications}
There are quite a few other standards besides patterns which can also
help software developers on improving the quality of the software. For
example as mentioned above, guidelines and examples. Others include,
but not limited to, principles, frameworks, and algorithms / data
structure. Some terms may have clashing definitions from the
human-computer interaction field, so it is a good idea we examine them
here. In this article, we propose the following classification for
these standards based on how specialized they are to the problems and
how concrete of the solutions they are offering (Figure
\ref{fig:standards}).

\begin{figure}[!t]
\centering
% Graphic for TeX using PGF
% Title: /home/ronhuang/Documents/Courses/NM5660/standards.dia
% Creator: Dia v0.97
% CreationDate: Tue May  4 00:16:22 2010
% For: ronhuang
% \usepackage{tikz}
% The following commands are not supported in PSTricks at present
% We define them conditionally, so when they are implemented,
% this pgf file will use them.
\ifx\du\undefined
  \newlength{\du}
\fi
\setlength{\du}{15\unitlength}
\begin{tikzpicture}
\pgftransformxscale{0.575520}
\pgftransformyscale{-0.575520}
\definecolor{dialinecolor}{rgb}{0.000000, 0.000000, 0.000000}
\pgfsetstrokecolor{dialinecolor}
\definecolor{dialinecolor}{rgb}{1.000000, 1.000000, 1.000000}
\pgfsetfillcolor{dialinecolor}
\pgfsetlinewidth{0.100000\du}
\pgfsetdash{}{0pt}
\pgfsetdash{}{0pt}
\pgfsetbuttcap
{
\definecolor{dialinecolor}{rgb}{0.000000, 0.000000, 0.000000}
\pgfsetfillcolor{dialinecolor}
% was here!!!
\pgfsetarrowsstart{stealth}
\pgfsetarrowsend{stealth}
\definecolor{dialinecolor}{rgb}{0.000000, 0.000000, 0.000000}
\pgfsetstrokecolor{dialinecolor}
\draw (5.000000\du,17.450000\du)--(30.700000\du,17.500000\du);
}
\pgfsetlinewidth{0.100000\du}
\pgfsetdash{}{0pt}
\pgfsetdash{}{0pt}
\pgfsetbuttcap
{
\definecolor{dialinecolor}{rgb}{0.000000, 0.000000, 0.000000}
\pgfsetfillcolor{dialinecolor}
% was here!!!
\pgfsetarrowsstart{stealth}
\pgfsetarrowsend{stealth}
\definecolor{dialinecolor}{rgb}{0.000000, 0.000000, 0.000000}
\pgfsetstrokecolor{dialinecolor}
\draw (17.450000\du,5.600000\du)--(17.450000\du,30.400000\du);
}
% setfont left to latex
\definecolor{dialinecolor}{rgb}{0.000000, 0.000000, 0.000000}
\pgfsetstrokecolor{dialinecolor}
\node at (17.400000\du,5.000000\du){Abstract};
% setfont left to latex
\definecolor{dialinecolor}{rgb}{0.000000, 0.000000, 0.000000}
\pgfsetstrokecolor{dialinecolor}
\node at (17.400000\du,31.100000\du){Concrete};
% setfont left to latex
\definecolor{dialinecolor}{rgb}{0.000000, 0.000000, 0.000000}
\pgfsetstrokecolor{dialinecolor}
\node at (29.150000\du,18.350000\du){Generalized};
% setfont left to latex
\definecolor{dialinecolor}{rgb}{0.000000, 0.000000, 0.000000}
\pgfsetstrokecolor{dialinecolor}
\node at (6.200000\du,18.450000\du){Specialized};
\definecolor{dialinecolor}{rgb}{1.000000, 1.000000, 1.000000}
\pgfsetfillcolor{dialinecolor}
\pgfpathellipse{\pgfpoint{18.875520\du}{7.846495\du}}{\pgfpoint{4.614020\du}{0\du}}{\pgfpoint{0\du}{1.153505\du}}
\pgfusepath{fill}
\pgfsetlinewidth{0.050000\du}
\pgfsetdash{}{0pt}
\pgfsetdash{}{0pt}
\pgfsetmiterjoin
\definecolor{dialinecolor}{rgb}{0.000000, 0.000000, 0.000000}
\pgfsetstrokecolor{dialinecolor}
\pgfpathellipse{\pgfpoint{18.875520\du}{7.846495\du}}{\pgfpoint{4.614020\du}{0\du}}{\pgfpoint{0\du}{1.153505\du}}
\pgfusepath{stroke}
% setfont left to latex
\definecolor{dialinecolor}{rgb}{0.000000, 0.000000, 0.000000}
\pgfsetstrokecolor{dialinecolor}
\node at (18.875520\du,8.041495\du){Principles};
\pgfsetlinewidth{0.050000\du}
\pgfsetdash{}{0pt}
\pgfsetdash{}{0pt}
\pgfsetmiterjoin
\pgfsetbuttcap
\definecolor{dialinecolor}{rgb}{1.000000, 1.000000, 1.000000}
\pgfsetfillcolor{dialinecolor}
\pgfpathmoveto{\pgfpoint{25.200000\du}{8.450000\du}}
\pgfpathcurveto{\pgfpoint{31.000000\du}{8.550000\du}}{\pgfpoint{28.450000\du}{20.650000\du}}{\pgfpoint{21.250000\du}{20.450000\du}}
\pgfpathcurveto{\pgfpoint{14.050000\du}{20.250000\du}}{\pgfpoint{19.400000\du}{8.350000\du}}{\pgfpoint{25.200000\du}{8.450000\du}}
\pgfusepath{fill}
\definecolor{dialinecolor}{rgb}{0.000000, 0.000000, 0.000000}
\pgfsetstrokecolor{dialinecolor}
\pgfpathmoveto{\pgfpoint{25.200000\du}{8.450000\du}}
\pgfpathcurveto{\pgfpoint{31.000000\du}{8.550000\du}}{\pgfpoint{28.450000\du}{20.650000\du}}{\pgfpoint{21.250000\du}{20.450000\du}}
\pgfpathcurveto{\pgfpoint{14.050000\du}{20.250000\du}}{\pgfpoint{19.400000\du}{8.350000\du}}{\pgfpoint{25.200000\du}{8.450000\du}}
\pgfusepath{stroke}
\definecolor{dialinecolor}{rgb}{1.000000, 1.000000, 1.000000}
\pgfsetfillcolor{dialinecolor}
\pgfpathellipse{\pgfpoint{24.213399\du}{11.734640\du}}{\pgfpoint{3.948799\du}{0\du}}{\pgfpoint{0\du}{1.502740\du}}
\pgfusepath{fill}
\pgfsetlinewidth{0.050000\du}
\pgfsetdash{}{0pt}
\pgfsetdash{}{0pt}
\pgfsetmiterjoin
\definecolor{dialinecolor}{rgb}{0.000000, 0.000000, 0.000000}
\pgfsetstrokecolor{dialinecolor}
\pgfpathellipse{\pgfpoint{24.213399\du}{11.734640\du}}{\pgfpoint{3.948799\du}{0\du}}{\pgfpoint{0\du}{1.502740\du}}
\pgfusepath{stroke}
% setfont left to latex
\definecolor{dialinecolor}{rgb}{0.000000, 0.000000, 0.000000}
\pgfsetstrokecolor{dialinecolor}
\node at (24.213399\du,11.529640\du){Architectural};
% setfont left to latex
\definecolor{dialinecolor}{rgb}{0.000000, 0.000000, 0.000000}
\pgfsetstrokecolor{dialinecolor}
\node at (24.213399\du,12.329640\du){Patterns};
\definecolor{dialinecolor}{rgb}{1.000000, 1.000000, 1.000000}
\pgfsetfillcolor{dialinecolor}
\pgfpathellipse{\pgfpoint{23.252014\du}{14.537121\du}}{\pgfpoint{4.197614\du}{0\du}}{\pgfpoint{0\du}{1.066921\du}}
\pgfusepath{fill}
\pgfsetlinewidth{0.050000\du}
\pgfsetdash{}{0pt}
\pgfsetdash{}{0pt}
\pgfsetmiterjoin
\definecolor{dialinecolor}{rgb}{0.000000, 0.000000, 0.000000}
\pgfsetstrokecolor{dialinecolor}
\pgfpathellipse{\pgfpoint{23.252014\du}{14.537121\du}}{\pgfpoint{4.197614\du}{0\du}}{\pgfpoint{0\du}{1.066921\du}}
\pgfusepath{stroke}
% setfont left to latex
\definecolor{dialinecolor}{rgb}{0.000000, 0.000000, 0.000000}
\pgfsetstrokecolor{dialinecolor}
\node at (23.252014\du,14.732121\du){Design Patterns};
\definecolor{dialinecolor}{rgb}{1.000000, 1.000000, 1.000000}
\pgfsetfillcolor{dialinecolor}
\pgfpathellipse{\pgfpoint{21.625000\du}{18.856017\du}}{\pgfpoint{2.825000\du}{0\du}}{\pgfpoint{0\du}{0.956017\du}}
\pgfusepath{fill}
\pgfsetlinewidth{0.050000\du}
\pgfsetdash{}{0pt}
\pgfsetdash{}{0pt}
\pgfsetmiterjoin
\definecolor{dialinecolor}{rgb}{0.000000, 0.000000, 0.000000}
\pgfsetstrokecolor{dialinecolor}
\pgfpathellipse{\pgfpoint{21.625000\du}{18.856017\du}}{\pgfpoint{2.825000\du}{0\du}}{\pgfpoint{0\du}{0.956017\du}}
\pgfusepath{stroke}
% setfont left to latex
\definecolor{dialinecolor}{rgb}{0.000000, 0.000000, 0.000000}
\pgfsetstrokecolor{dialinecolor}
\node at (21.625000\du,19.051017\du){Idioms};
% setfont left to latex
\definecolor{dialinecolor}{rgb}{0.000000, 0.000000, 0.000000}
\pgfsetstrokecolor{dialinecolor}
\node at (24.750000\du,9.450000\du){Patterns};
\definecolor{dialinecolor}{rgb}{1.000000, 1.000000, 1.000000}
\pgfsetfillcolor{dialinecolor}
\pgfpathellipse{\pgfpoint{12.800000\du}{15.550000\du}}{\pgfpoint{4.000000\du}{0\du}}{\pgfpoint{0\du}{1.000000\du}}
\pgfusepath{fill}
\pgfsetlinewidth{0.050000\du}
\pgfsetdash{}{0pt}
\pgfsetdash{}{0pt}
\pgfsetmiterjoin
\definecolor{dialinecolor}{rgb}{0.000000, 0.000000, 0.000000}
\pgfsetstrokecolor{dialinecolor}
\pgfpathellipse{\pgfpoint{12.800000\du}{15.550000\du}}{\pgfpoint{4.000000\du}{0\du}}{\pgfpoint{0\du}{1.000000\du}}
\pgfusepath{stroke}
% setfont left to latex
\definecolor{dialinecolor}{rgb}{0.000000, 0.000000, 0.000000}
\pgfsetstrokecolor{dialinecolor}
\node at (12.800000\du,15.745000\du){Guidelines};
\definecolor{dialinecolor}{rgb}{1.000000, 1.000000, 1.000000}
\pgfsetfillcolor{dialinecolor}
\pgfpathellipse{\pgfpoint{8.107896\du}{26.291354\du}}{\pgfpoint{2.897916\du}{0\du}}{\pgfpoint{0\du}{1.077854\du}}
\pgfusepath{fill}
\pgfsetlinewidth{0.050000\du}
\pgfsetdash{}{0pt}
\pgfsetdash{}{0pt}
\pgfsetmiterjoin
\definecolor{dialinecolor}{rgb}{0.000000, 0.000000, 0.000000}
\pgfsetstrokecolor{dialinecolor}
\pgfpathellipse{\pgfpoint{8.107896\du}{26.291354\du}}{\pgfpoint{2.897916\du}{0\du}}{\pgfpoint{0\du}{1.077854\du}}
\pgfusepath{stroke}
% setfont left to latex
\definecolor{dialinecolor}{rgb}{0.000000, 0.000000, 0.000000}
\pgfsetstrokecolor{dialinecolor}
\node at (8.107896\du,26.486354\du){Examples};
\definecolor{dialinecolor}{rgb}{1.000000, 1.000000, 1.000000}
\pgfsetfillcolor{dialinecolor}
\pgfpathellipse{\pgfpoint{17.448100\du}{24.734700\du}}{\pgfpoint{5.261200\du}{0\du}}{\pgfpoint{0\du}{1.315300\du}}
\pgfusepath{fill}
\pgfsetlinewidth{0.050000\du}
\pgfsetdash{}{0pt}
\pgfsetdash{}{0pt}
\pgfsetmiterjoin
\definecolor{dialinecolor}{rgb}{0.000000, 0.000000, 0.000000}
\pgfsetstrokecolor{dialinecolor}
\pgfpathellipse{\pgfpoint{17.448100\du}{24.734700\du}}{\pgfpoint{5.261200\du}{0\du}}{\pgfpoint{0\du}{1.315300\du}}
\pgfusepath{stroke}
% setfont left to latex
\definecolor{dialinecolor}{rgb}{0.000000, 0.000000, 0.000000}
\pgfsetstrokecolor{dialinecolor}
\node at (17.448100\du,24.929700\du){Frameworks};
\definecolor{dialinecolor}{rgb}{1.000000, 1.000000, 1.000000}
\pgfsetfillcolor{dialinecolor}
\pgfpathellipse{\pgfpoint{14.607469\du}{21.333937\du}}{\pgfpoint{4.692529\du}{0\du}}{\pgfpoint{0\du}{1.666037\du}}
\pgfusepath{fill}
\pgfsetlinewidth{0.050000\du}
\pgfsetdash{}{0pt}
\pgfsetdash{}{0pt}
\pgfsetmiterjoin
\definecolor{dialinecolor}{rgb}{0.000000, 0.000000, 0.000000}
\pgfsetstrokecolor{dialinecolor}
\pgfpathellipse{\pgfpoint{14.607469\du}{21.333937\du}}{\pgfpoint{4.692529\du}{0\du}}{\pgfpoint{0\du}{1.666037\du}}
\pgfusepath{stroke}
% setfont left to latex
\definecolor{dialinecolor}{rgb}{0.000000, 0.000000, 0.000000}
\pgfsetstrokecolor{dialinecolor}
\node at (14.607469\du,21.128937\du){Algorithms /};
% setfont left to latex
\definecolor{dialinecolor}{rgb}{0.000000, 0.000000, 0.000000}
\pgfsetstrokecolor{dialinecolor}
\node at (14.607469\du,21.928937\du){Data structures};
\end{tikzpicture}

\caption{Taxonomy of software development standards.}
\label{fig:standards}
\end{figure}

\textit{Patterns} are split into three kinds in figure
\ref{fig:standards}. The differences between the three are in their
corresponding levels of abstraction and details. \textit{Architectural
  patterns} involves with high-level components and mechanisms of the
software. They also have wider affect on the overall software
design. \textit{Design Patterns} are more related to medium-level
subsystems within the software and have no influence on the overall
design. \textit{Idioms} are paradigm-specific programming language
techniques for low-level components \citep{patterns:buschmann}.

\textit{Principles} do not describe the solutions to the encountered
problems and are more abstract than patterns
\citep{patterns:coplien}. However, principles frequently become the
force and/or rationale of a patterns \citep{patterns:appleton}.

\textit{Guidelines} are also very visible in the software development
domain. Almost every programming language has its own programming
style guidelines, which suggest to the software developers the style
on how the source code should be organized, how variables should be
name, etc. Similarly, almost every major operating system has its own
human interface guidelines, which give recommendations to the software
developers on improving the usability of the end-users. However,
guidelines largely do not emphasize on the problem nor on the process
of developing and using the guidelines. Guidelines normally just give
``imperative'' directions to the software developers and only apply to
a specific product or system \citep{patterns:griffiths}.

\textit{Frameworks} are also closely related to patterns. Frameworks
typically encompass several patterns, or implementations of a system
of patterns. However, frameworks are more concrete than patterns in
the sense that they can be embodied in code and can be reused
directly. They are executable software, whereas patterns are knowledge
and experience about software. Also frameworks always have a specific
application domain, while patterns are applicable to nearly any kind
of application \citep{patterns:gamma}.

\textit{Algorithms / data structures} generally solve more
fine-grained computational problems or deal with resource
optimizations. Patterns on the other hand deal with software
architectural issues \citep{patterns:appleton}.

\textit{Examples} are perhaps the most effective and direct among all
the above. Almost all software developers learn new programming
languages or new techniques through examples. One classic example is
the ``Hello World'' program \citep{c:kernighan}, which demonstrates a
simple way of displaying the text ``Hello World'' with a specific
programming language. However examples are also the most restrictive,
providing very precised solutions to specific settings, and the most
concrete, providing mainly source code, of all standards.


Now that a clearer understanding of what patterns are referring to in
this article, we now move our attention to user experience and
transparency. We talk about their development from the early user
interface designs to the more recent advancements.

\subsection{Technology-oriented Development}
When junior software developers are given the task of designing a new
product or system, most of the time, they tend to squeeze as many
features as possible into whatever they are suppose to design. The
interfaces between the user and the design are scattered all
over. This can often be observed in corporates where a junior software
developer is assigned to develop an internal tool to facilitate an
existing working process. Though the design is practical, it is not
very useful and the usage is limited to only a small selected group of
users.

The above phenomenons were also observed in the early days of user
interface designs. Many new technologies---such as window, icon, menu,
and pointing device---had grasped the attentions of the software
developers. They were very eager to try out these new technologies and
often falsely believed that the implementation of these new
technologies were the goal of successful software.

This is indicated by \citet{inmates:cooper} that although software
developers work hard to make their software easy to use, their frame
of reference is themselves and as a result they make it easy for other
software developers, instead of typical end-users. He explains that
since having too much influence over the design of the human interface
and the lack of skills in this area, software developers do a poor job
of it.

\subsection{Usability}
This situation will last until the software developers are introduced
to the concept of usability. Since the 1980s, this concept was
introduced by the published studies and analysis papers from a
dedicated group of mostly psychologists and human factors researchers
\citep{human:rubinstein, friendly:simpson, human:shneiderman,
  human:brown, software:dumas}. The philosophy behind this is that
every stage of the development process---requirements, design,
implementation, verification, and maintenance---are given great
attention to the needs and limitations of the users.

In this user-oriented development, contrast to the previous
technology-oriented or feature-oriented development, software
developers work with experts from various specific fields---and most
importantly, also collaborate closely with the actual users---to bring
efficient, easy-to-learn, easy-to-memorize, and non-disruptive product
to the user. They form psychology and physiology models of the user to
anticipate their needs and limitations. And from that model, the
software developers could build products that perfectly fit the users
and not get in the way. Where the users can focus their attention on
the tasks at hand as if the user interfaces do not exist or seem
transparent \citep{computer:weiser}.

This interface transparency is an indication of good user interface
design. From their research, \citet{transparency:holtzblatt} show that
elements of an application design can disrupt users' work. They
observed that only when the users are not disrupted by the computer
system will they remain in the flow of their work and experience
interface transparency. It is also one of the ultimate goals for
usability studies and is considered the ideal relationship between
user and tool with the tool seeming to disappear by
\citet{transparency:rutkoski}.

\subsection{User Experience}
However, the idea of striving for interface transparency is at the
same time being challenged by others. \citet{windows:bolter} talked
about the myth of transparency: \textit{The danger of transparency is
  that the interface will mask the operation of the system exactly
  when the user needs to see and understand what the system is
  doing}. One reason for this danger comes from the fact that
designers and the users each have their own conceptual models or
mental images of the product. Ideally, these two models should be
identical for the users to understand and use the product
properly. Unfortunately, this might not always be the case. So the
users often have to form their own model exclusively from the
observation of the product, whether from the exterior guise, the
feedback provided, the responses from blogger, documentations, etc
\citet{design:norman}. But if the product was invisible or transparent
to the users, how could it act as a medium to help the users to
understand itself? Interface transparency eliminates the communicative
aspect of the user interfaces, along with others such as the engaging
aspect.

% other direction: user experience
% difficulty of user experience
The community of human-computer interaction quickly embraced the idea
that solely pursuing for interface transparency and performance is not
enough, and a richer model is required to encapsulate the other
perspectives of the user interaction. These all fall under the
umbrella term of user experience, which may include: addressing the
human needs beyond the ``functional''; stressing affective and
emotional aspects of the interaction; and dealing with the nature of
experience \citep{ux:hassenzahl}. Many models have been proposed,
however the universal definition for user experience is still not well
understood nor fully clarified. Some definitions of user experience
are included here to illustrate its diversity:
\begin{quote}
  {\it ``All the aspects of how people use an interactive product: the
    way it feels in their hands, how well they understand how it
    works, how they feel about it while they're using it, how well it
    serves their purposes, and how well it fits into the entire
    context in which they are using it.''}

  \raggedleft \citet{experience:alben}
\end{quote}

\begin{quote}
  {\it ``All aspects of the end-user's interaction with the company,
    its services, and its product.''}

  \raggedleft \citet{experience:nielsen}
\end{quote}

\begin{quote}
  {\it ``A consequence of a user's internal state (predispositions,
    expectations, needs, motivation, mood, etc.), the characteristics
    of the designed system (e.g. complexity, purpose, usability,
    functionality, etc.) and the context (or the environment) within
    which the interaction occurs (e.g. organizational/social setting,
    meaningfulness of the activity, voluntariness of use, etc.)''}

  \raggedleft \citet{ux:hassenzahl}
\end{description}

There are several reasons why this is so, according to \citet{ux:law}
for example:
\begin{enumerate}
  \item User experience is associated with a wide range of vague and
    dynamic concepts, including emotional, affective, experiential,
    hedonic, and aesthetic variables. Whether or not to include each
    of these variables is greatly dependent on the interests and
    background of the proposing author.
  \item The unit for measuring user experience is
    \textit{malleable}. Unlike usability where only a single aspect of
    interaction between the user and the product is assessed with well
    defined criteria, it could be extended to include multiple users,
    environment while the product is being used, and more.
\end{enumerate}


\subsection{Transparency}
[FIXME]


\section{Patterns Of User Experience and Transparency}
\label{sec:pux}
Patterns of user experience, using the language similar to that of
software patterns, describe a set of standards that help software
developers on providing good user experience to the end-users with
their software. From the previous practices of and knowledge gained by
implementing software patterns, patterns of user experiences provide a
familiar sense to software developers. They are perhaps what the
software developers really need to improve the user experience of the
typical end-users, instead of the currently yet clarified definitions
and statements of user experience, and are perhaps also the one
crucial factor for this to succeed.

\subsection{For End-users}
Interestingly, the traditional software patterns can also be
considered as patterns of user experience \citep{pux:blackwell}. But
they are referring to user experience for the \textit{software
  developers}, i.e.\ software developers will have better experience
with the source code if they choose to implement the appropriate
software patterns into their design. For example, following a well
known software pattern can increase the maintainability of the source
code among the collaborating software developers.

But this intention for software patterns differs from the original
intention traced back since architecture and urban planning
\citep{timeless:alexander}. The original intention was to improve
comfort and quality of life for the people living inside. But due to
an early, albeit still existing, common misconception among software
developers, the intention of patterns turned in the wrong direction
when introduced to the software development domain. They often focus
too much attention on engineering the user interface and emphasizing
the technical details that, instead as supporting tools, these user
interface and technical details become the final product for the users
\citep{pux:blackwell}. That is why software patterns are more
applicable toward software developers.

However, the user experience that we refer to is the user experience
for the \textit{end-users}, the actual users who will be using the
software. Software developers are still the ones who have to implement
the patterns of user experience. But it is the user experiences with
the product of the end-users that are improved, instead of the user
experiences with the source code of the software developers.

\subsection{Examples}
Table \ref{tab:shield} gives a simple example on patterns of user
experience for error management \citep{patterns:welie}. Coherent to
the definition in section \ref{sec:pattern}, the pattern clearly
specified the problem, the context, the forces, and the
solutions. Some additional fields are also included to help people
understanding the pattern better.

For comparison, the following is a user interface guideline on how
text should be formatted: \textit{The \textbf{system font} (Lucida
  Grande Regular 13 point) is used for text in menus, dialogs, and
  full-size controls} \citep{hig:apple}. Contrast to patterns, it does
not provide the ins and outs on how the solution was devised. And
although it offers a direct solution, it is very specific to a
particular problem.

\begin{table}[!t]
  \caption{Pattern of user experience `The Shield'.}
  \label{tab:shield}
  \begin{center}
    \begin{tabular}{| p{0.25\columnwidth} || p{0.65\columnwidth} |}
      \hline
      Name & The Shield. \\ \hline

      Problem & The user may accidentally select a function that has
      irreversible (side) effects. \\ \hline

      Usability Principle & Error management. \\ \hline

      Context & The user needs to be protected against unintended or
      accidental actions that have irreversible (side) effects. The
      (side) effects may lead to unsafe or highly undesired
      situations. For example the unintended deletion or overwriting
      of files. Do not use for actions that are reversible. \\ \hline

      Forces & The user is striving for speed while trying to avoid
      mistakes. \\ \hline

      Solutions & Protect the user by inserting a shield. \\ \hline

      Usability Impact & Increased safety, less errors and higher
      satisfaction. However, it requires extra user action which leads
      to lower performance time. \\ \hline

      Rationale & The extra layer causes the user to require 2
      repetitive mistakes instead of 1. The safe default decreases the
      chances for a second mistake. \\ \hline

      Known uses & Microsoft Explorer, Apple Finder. \\ \hline
    \end{tabular}
  \end{center}
\end{table}

\subsection{For Transparency}
% Talk about various other non interface transparency.
% Talk about Linux and Windows as example for transparency.

% * Different meanings of interface transparency.
% * Brief mentioning of interface transparency in the history of
%   human-computer interaction.
[FIXME]

Though there are quite a few existing publications discussing patterns
of user experience in various aspects of human-computer interaction
(refer to section \ref{sec:related}), we will focus our attention on
investigating the patterns of user experience for interface
transparency.

Interface transparency often have different meanings when presented to
diverse set of audiences. For expert system, interface transparency is
important for gaining trust to the users [FIXME].

Similarly, for Unix power users following the Unix philosophy,
interface transparency might suggest that the software is capable of
churning out excessive amount of messages when error occurred. These
power users are able to interpret the messages and partially `replay'
the flow of control\footnote{Refers to the order in which the
  individual statements or instructions of a program are executed or
  evaluated.}  for the software. The software at fault is not a
completely opaque black box to them with the help of the
messages. They can sort of see through the software and figure out the
problem hiding underneath \citep{unix:raymond}.

On the other hand, typical end-users do not usually consider excessive
messages as interface transparency. A common action when typical
end-users encounter unexpected information: they try to get rid of it
as soon as possible \citep{oldnew:chen}. To them, the information are
regarded as either gibberish or elaborated way of describing something
dangerous. This is because the information is unforeseen and cannot
fit into the conceptual model which they had previously formed for the
software. Hence, instead of being treated as interface transparency,
the excessive messages are more like an unknown dark spot occluding
the flow of experience\footnote{Refers to the mental state of
  operation in which a person is fully immersed in a feeling of full
  involvement in the process of an activity.}.


% Form pattern and guidelines for software developer, with respect to
% interface transparency.
[Elaborated paragraphs explaining interface transparency as my second
  research interest.]

Added to this, the emergence of the user experience is an attempt to
inform design and evaluation has been widespread in HCI. Such that,
researchers are proposing user experience encapsulating the idea of
interface transparency.

Hence, the relationship between transparency and user
experience will also be examined.


\section{Planned Experiments}
\label{sec:experiment}
[FIXME]

\subsection{Research Goal}
[Elaborated paragraphs combining patterns of user experience and
  interface transparency as my research goal.]

% problem I am trying to solve
With all this being said, this article is not another attempt to
develop a common view on user experience. Numerous literature,
conferences, and workshops have already attempted to address this
topic, such as \citet{early:forlizzi}, \citet{emotional:norman},
\citet{action:dourish}, \citet{ux:hassenzahl},
\citet{experience:desmet}, and \citet{ux:law}.

For sake of completeness, some view points on user experience will
still be discussed in the following section. However, the main
research goal for this article is to further explore interface
transparency within the field of user experience. Interface
transparency is still a crucial factor in this field, even though user
experience has been the prominent subject in recent years. In his
article, \citet{future:memmel} suggests that usability engineering and
interaction design will be the more concrete premises and disciplines
below the umbrella term of user experience. This argument is similar
to the one from \citet{windows:bolter} that transparency is only half
the story---each digital design should move back and forth between
being transparent and being reflective.

Transparent and reflective, two contrasting albeit interesting point
of views prompted people to think about whether they have been too
obsessed with achieving interface transparency, the benefits and
limitations of interface transparency, measuring interface
transparency. Added to this, the emergence of the user experience is
an attempt to inform design and evaluation has been widespread in
HCI. Such that, researchers are proposing user experience
encapsulating the idea of transparency. Hence, the relationship
between transparency and user experience will also be examined.

% Talk about the experiment.
\subsection{Experiment}
[FIXME]

A new human interface device and application will be developed to
investigate interface transparency within user experience. This new
device will provide similar functionality to the multi-touch touchpad
that comes with recent laptops. However, the device will enable the
users to perform the usual tasks of multi-touch touchpad on any
surface, as if the physical touchpad is underneath their hand. The
development of this new device is actually based upon our previous
study \citep{lmnt:huang}, where the first-generation prototype was
built and some preliminary studies were conducted. Much of the
feedback from the first-generation prototype are taken into
consideration when designing the new device for this study.

% Application based on Baba Painter.
An application will be developed to accompany the new human interface
device. This application will too be an extension to our other
previous study, an interactive painting and collaboration reviewing
system \citep{baba:abeyrathne}. However, since the instrument will be
replaced by the new device developed in this study, the scenario and
the users will be adjusted accordingly. We planed to target the
application to kids. The application will allow small children to
scribble intuitively on any surface and be creative wherever they are,
whether on the dining table of a restaurant or on the wall in the
bedroom, while without causing any mess or damage.

% User, product, context


\section{Conclusion}
\label{sec:conclusion}
[FIXME]


\bibliographystyle{apalike}
\bibliography{final}


\end{document}
