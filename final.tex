% final.tex
% Final Report for NM5660 Independent Study Module
% Author: Yih-Lun Huang
% Revisions: 16 April 2010

\documentclass{acm_proc_article-sp}

\usepackage{verbatim}
\usepackage{natbib}
\usepackage{url}
\usepackage{tikz}

\hyphenation{trans-pa-ren-cy}

\begin{document}

% TODO: change to ``Toward...''?
\title{Investigation of the relationship between transparency and user
  experience}

\numberofauthors{1}
\author{
\alignauthor Yih-Lun Huang\\
\affaddr{HY090191Y}\\
\email{ylhuang@nus.edu.sg}
}

\maketitle
\begin{abstract}
[FIXME]
\end{abstract}


\category{H.5.2}{Information Interfaces And Presentation}{User
  Interface}[input devices and strategies]

\terms{Design, Algorithms, Measurement}

\keywords{Input device, multi-touch, touchpad, lazy}


\section{Introduction}
The notion of user experience has been wildly adopted in the field of
human-computer interaction. However, as many existing literature
pointed out already \citep{ux:hassenzahl, ux:law}, a common agreement
on the exact definition of user experience is still in debate, and is
so far not fully understood by the people---researchers, consultants,
managers in the industry, and practitioners---in the field.

They do agree that the traditional usability framework, which focus
primarily on work-related and performance-based use, is limited and
can no longer model the ever more complex settings of the modern
computation. This was especially true when ubiquitous computing was
introduced, where the interfaces between users and computers are no
longer single, static, and restricted to the traditional window, icon,
menu, and pointing device. Instead, multiple interfaces are spread
throughout the environment and are updated dynamically depending on
the context \citep{windows:bolter}.

But besides the above agreement, people from different backgrounds and
interests, with varying focuses and concerns, have formulated their
own definitions and statements on user experience. Some tackle this
from the emotional and affective aspects of human-computer
interaction, such as stressing the importance of emotions as
consequences of product use \citep{emotions:desmet}. Some deal with
the nature of experience or the experiential aspect,
e.g.\ \citet{experience:forlizzi}. And yet others address the human
needs beyond the instrumental, such as surprise or intimacy.
\citep{alternatives:gaver}.

Though very exciting to witness and participate in this ``disordered''
time frame, where opportunities are still around to make significant
contributions to the human knowledge, it is actually quite troublesome
for practitioners in the field of human-computer interaction. And in
the case of this article, we specifically focus our attention on
software developers.

Among human-computer interaction practitioners from all walks of life,
software developers are of special interest to us: software developers
have relatively higher percentage than other professions in the
information technology industry. In most economies, small and medium
enterprises are much greater in number. For example, they comprise
approximately 99\% of all firms in European
Union\footnote{\url{http://epp.eurostat.ec.europa.eu/portal/page/portal/european_business/special_topics/small_medium_sized_enterprises_SMEs}}. And
they tend to employ more personnel closer to the core business they
are providing, due to the smaller size and limited budgets. So for
small and medium enterprises which are related to software or
human-computer interaction, software developers more often than not
will have higher percentage in the demographic profiles, not designers
nor user experience engineers. Hence in order to effectively improve
the \textit{user experience} of human-computer interaction, we need
effective measures for software developers to implement it into their
product.

In the following sections, we will explain why patterns of user
experience are an effective measure to improve user experience and
also the particular subject (interface transparency) in user
experience we are interested in (section \ref{sec:pux}). Then we will
cover a review on the various researches that led to the current state
of user experience and interface transparency (section
\ref{sec:ux}). Next, a proposal for experiments will be given to
explore the possible patterns of user experience (section
\ref{sec:experiment}). Finally, the conclusion (section
\ref{sec:conclusion}).


\section{Related Works}
\label{sec:relatedworks}
% TODO: mention Universal usability
[FIXME]

Many patterns of user experience from various facets of human-computer
interaction are also being published or discussed.  For instance,
\citet{patterns:tidwell} discussed about general subjects in
human-computer interaction; \citet{touch:boudreaux} talked exclusively
on user experience for multi-touch input device;
\citet{participatory:dearden} examined the patterns for enabling users
to participate in the product design.

\citet{unix:raymond}


\section{Patterns of User Experience}
\label{sec:pux}
Software developers, like engineers in the other engineering
disciplines, rely on standards---codified and enforced from good
practices---to improve the quality and lower the cost of software
development \citep{practice:ipenz}. These standards can be in the form
of examples, guidelines or even abstractified into patterns.

Patterns, as described by \citet{patterns:griffiths}, are abstractions
from the specific examples, which give patterns their generative
power. They do not supply immediate solutions to the problems,
however. Creativity is required for the software developers to
implement the patterns. Patterns are also considered more advantageous
than guidelines because they are more related to a context and are
problem centered \citep{patterns:welie}.

Software patterns are of great values to software developers. They
provided valuable engineering solutions to difficult technical
problems or elegant tricks to build user interfaces. Mastering
software patterns is so influential that it is often regarded as an
indication to distinguishing rockstar software developers from
mediocre ones \citep{rockstar:iskold}. But what are patterns?

\subsection{Patterns}
The term ``pattern'' originated from \citet{timeless:alexander}, where
the author was looking for structures of urban planning and building
architecture that have positive influence on people by improving their
comfort and quality of life. He described:
\begin{quote}
Each pattern is a three-part rule, which expresses a relation between
a certain context, a problem, and a solution.

As an element in the world, each pattern is a relationship between a
certain context, a certain system of forces which occurs repeatedly in
that context, and a certain spatial configuration which allows these
forces to resolve themselves.

As an element of language, a pattern is an instruction, which shows
how this spatial configuration can be used, over and over again, to
resolve the given system of forces, wherever the context makes it
relevant.

The pattern is, in short, at the same time a thing, which happens in
the world, and the rule which tells us how to create that thing, and
when we must create it. It is both a process and a thing; both a
description of a thing which is alive, and a description of the
process which will generate that thing.
\end{quote}

Several years later, this idea of patterns was introduced to the
software development domain by \citet{patterns:beck} for designing
user interfaces with Smalltalk. This eventually led to one of the most
important publications in software development, \textit{Design
  Patterns: Elements of Reusable Object-Oriented Software} from
\citet{patterns:gamma}.

Though many have given their own definitions of patterns from the
software perspective, most of them still closely resemble to the
original definition from Alexander. For example,
\citet{patterns:appleton} gave:
\begin{quote}
A pattern is a named nugget of instructive information that captures
the essential structure and insight of a successful family of proven
solutions to a recurring problem that arises within a certain context
and system of forces.
\end{quote}
And \citet{timeless:gabriel} gave:
\begin{quote}
Each pattern is a three-part rule, which expresses a relation between
a certain context, a certain system of forces which occurs repeatedly
in that context, and a certain software configuration which allows
these forces to resolve themselves.
\end{quote}

\subsection{Comparison}
[FIXME: add diagram.]

\subsection{Of User Experience}
Analogously, patterns of user experience---a language for describing
user experiences with structured information
\citep{pux:blackwell}---is the crucial factor for the success of
bringing better user experience to the users. This is what
practitioners like software developers want, instead of the exact
definitions and statements of user experience.

Interestingly, the traditional software patterns can also be
considered as patterns of user experience. But they are referring to
user experience for the \textit{software developers}, i.e.\ software
developers will have better experience with the source code if they
choose to implement appropriate software patterns into their
design. For example, following a well known software pattern can
increase the maintainability of the source code. Notice that this also
reflects an early, albeit still existing, common misconception of
software developers \citep{pux:blackwell}. They often focus too much
attention on engineering the user interface and emphasizing the
technical details that, instead as supporting tools, these user
interface and technical details become the final product for the
users.

\subsection{Examples}
However the user experience we are mentioning here is the user
experience for the \textit{users}, the actual users who will be using
the software. By implementing these patterns of user experience into
the design, the users can gain better user experience with the
product. Table~\ref{tab:shield} gives a simple example on patterns of
user experience for error management \citep{patterns:welie}.

\begin{table}[!t]
  \caption{Pattern of user experience `The Shield'.}
  \label{tab:shield}
  \begin{center}
    \begin{tabular}{| p{0.2\columnwidth} || p{0.7\columnwidth} |}
      \hline
      Name & The Shield. \\ \hline

      Problem & The user may accidentally select a function that has
      irreversible (side) effects. \\ \hline

      Usability Principle & Error management. \\ \hline

      Context & The user needs to be protected against unintended or
      accidental actions that have irreversible (side) effects. The
      (side) effects may lead to unsafe or highly undesired
      situations. For example the unintended deletion or overwriting
      of files. Do not use for actions that are reversible. \\ \hline

      Forces & The user is striving for speed while trying to avoid
      mistakes. \\ \hline

      Solutions & Protect the user by inserting a shield. \\ \hline

      Usability Impact & Increased safety, less errors and higher
      satisfaction. However, it requires extra user action which leads
      to lower performance time. \\ \hline

      Rationale & The extra layer causes the user to require 2
      repetitive mistakes instead of 1. The safe default decreases the
      chances for a second mistake. \\ \hline

      Known uses & Microsoft Explorer, Apple Finder. \\ \hline
    \end{tabular}
  \end{center}
\end{table}

\subsection{For Interface Transparency}
% * Different meanings of interface transparency.
% * Brief mentioning of interface transparency in the history of
%   human-computer interaction.

Though there are quite a few existing publications discussing patterns
of user experience in various aspects of human-computer interaction
(refer to section \ref{sec:relatedworks}), we will focus our attention
on investigating the patterns of user experience for interface
transparency.

Interface transparency often have different meanings when presented to
diverse set of audiences. For expert system, interface transparency is
important for gaining trust to the users [FIXME].

Similarly, for Unix power users following the Unix philosophy,
interface transparency might suggest that the software is capable of
churning out excessive amount of messages when error occurred. These
power users are able to interpret the messages and partially `replay'
the flow of control\footnote{Refers to the order in which the
  individual statements or instructions of a program are executed or
  evaluated.}  for the software. The software at fault is not a
completely opaque black box to them with the help of the
messages. They can sort of see through the software and figure out the
problem hiding underneath \citep{unix:raymond}.

On the other hand, typical end-users do not usually consider excessive
messages as interface transparency. A common action when typical
end-users encounter unexpected information: they try to get rid of it
as soon as possible \citep{oldnew:chen}. To them, the information are
regarded as either gibberish or elaborated way of describing something
dangerous. This is because the information is unforeseen and cannot
fit into the conceptual model which they had previously formed for the
software. Hence, instead of being treated as interface transparency,
the excessive messages are more like an unknown dark spot occluding
the flow of experience\footnote{Refers to the mental state of
  operation in which a person is fully immersed in a feeling of full
  involvement in the process of an activity.}.


% Form pattern and guidelines for software developer, with respect to
% interface transparency.
[Elaborated paragraphs explaining interface transparency as my second
  research interest.]

Added to this, the emergence of the user experience is an attempt to
inform design and evaluation has been widespread in HCI. Such that,
researchers are proposing user experience encapsulating the idea of
interface transparency.

Hence, the relationship between transparency and user
experience will also be examined.

\subsection{Research Goal}
[Elaborated paragraphs combining patterns of user experience and
  interface transparency as my research goal.]

% problem I am trying to solve
With all this being said, this article is not another attempt to
develop a common view on user experience. Numerous literature,
conferences, and workshops have already attempted to address this
topic, such as \citet{early:forlizzi}, \citet{emotional:norman},
\citet{action:dourish}, \citet{ux:hassenzahl},
\citet{experience:desmet}, and \citet{ux:law}.

For sake of completeness, some view points on user experience will
still be discussed in the following section. However, the main
research goal for this article is to further explore interface
transparency within the field of user experience. Interface
transparency is still a crucial factor in this field, even though user
experience has been the prominent subject in recent years. In his
article, \citet{future:memmel} suggests that usability engineering and
interaction design will be the more concrete premises and disciplines
below the umbrella term of user experience. This argument is similar
to the one from \citet{windows:bolter} that transparency is only half
the story---each digital design should move back and forth between
being transparent and being reflective.

Transparent and reflective, two contrasting albeit interesting point
of views prompted people to think about whether they have been too
obsessed with achieving interface transparency, the benefits and
limitations of interface transparency, measuring interface
transparency. Added to this, the emergence of the user experience is
an attempt to inform design and evaluation has been widespread in
HCI. Such that, researchers are proposing user experience
encapsulating the idea of transparency. Hence, the relationship
between transparency and user experience will also be examined.

% Talk about the experiment.
\subsection{Experiment}
[FIXME]

A new human interface device and application will be developed to
investigate interface transparency within user experience. This new
device will provide similar functionality to the multi-touch touchpad
that comes with recent laptops. However, the device will enable the
users to perform the usual tasks of multi-touch touchpad on any
surface, as if the physical touchpad is underneath their hand. The
development of this new device is actually based upon our previous
study \citep{lmnt:huang}, where the first-generation prototype was
built and some preliminary studies were conducted. Much of the
feedback from the first-generation prototype are taken into
consideration when designing the new device for this study.

% Application based on Baba Painter.
An application will be developed to accompany the new human interface
device. This application will too be an extension to our other
previous study, an interactive painting and collaboration reviewing
system \citep{baba:abeyrathne}. However, since the instrument will be
replaced by the new device developed in this study, the scenario and
the users will be adjusted accordingly. We planed to target the
application to kids. The application will allow small children to
scribble intuitively on any surface and be creative wherever they are,
whether on the dining table of a restaurant or on the wall in the
bedroom, while without causing any mess or damage.

% User, product, context


\section{User Experience}
\label{sec:ux}
[FIXME]

When junior software developers are given the task of designing a new
product or system\footnote{In the following text, ``product'' is used
  as a general term to refer to a product, system, service, or
  non-commercial item, unless explicitly specified otherwise.}, most
of the time, they tend to squeeze as many features as possible into
whatever they are suppose to design. The interfaces between the user
and the design are scattered all over. This can often be observed in
corporates where a junior software developer is assigned to develop an
internal tool to facilitate an existing working process
(Figure~\ref{fig:featureful}.)  Though the design is practical, it is
not very useful and the usage is limited to only a small selected
group of users.

\begin{figure}[!t]
\centering
\includegraphics[width=.7\columnwidth]{featureful}
\caption{Internal tool for parsing the content in RAM.}
\label{fig:featureful}
\end{figure}

Software developers have weird sense of excitement when they acquire
new technologies and are often very eager to try them out. Lack of
proper testing platform, many software developers simply test the newly
acquired technology on whatever task they have at hand. Since efforts
and time are already spent on making the new technology functional and
it also provides a ``cool'' new feature to the original product, the
test is not removed afterward but instead becomes part of the product.

This is indicated by Alan Cooper \citeyearpar{inmates:cooper} that
although software developers work hard to make their software easy to
use, their frame of reference is themselves and as a result they make
it easy for other software developers, instead of typical end-users. He
explains that since having too much influence over the design of the
human interface and the lack of skills in this area, software
developers do a poor job of it.

\subsection{Usability and Interface Transparency}
% TODO: talk about user-centered design and usability
This phenomenon will last until the software developers are introduced
to the concept of usability. Dated back since the 1980s, this concept
was introduced by the published studies and analysis papers from a
dedicated group of mostly psychologists and human factors researchers
\citep{human:rubinstein, friendly:simpson, human:shneiderman,
  human:brown, software:dumas}. The philosophy behind this is that
every stage of the development process---requirements, design,
implementation, verification, and maintenance---are given great
attention to the needs and limitations of the users.

In this user-oriented development, contrast to the previous
technology-oriented or feature-oriented development, software
developers work with experts from various specific fields---and most
importantly, also collaborate closely with the actual users---to bring
efficient, easy-to-learn, easy-to-memorize, and non-disruptive product
to the user. They form psychology and physiology models of the user to
anticipate their needs and limitations. And from that model, the
software developers can build the product that would perfectly fit the
users.

% Bring up transparency, disrupt
From their research, \citet{transparency:holtzblatt} show that
elements of an application design can disrupt users' work. They
observed that only when the users are not disrupted by the computer
system will they remain in the flow of their work and experience
interface transparency. This interface transparency is the ultimate
goal for usability studies and is considered the ideal relationship
between user and tool with the tool seeming to disappear by
\citet{transparency:rutkoski}.

\subsection{Danger of Transparency}
% danger of transparency
However, the idea of striving for interface transparency is also
challenged by others. \citet{windows:bolter} talked about the myth of
transparency: \textit{The danger of transparency is that the interface
  will mask the operation of the system exactly when the user needs to
  see and understand what the system is doing}. As described by
\citet{design:norman}, the designers and the users each have their own
conceptual models or mental images of the product. Ideally, these two
models should be identical for the users to understand and use the
product properly. Unfortunately, this might not always be the case. So
the users often have to form their own model exclusively from the
observation of the product, whether from the exterior guise, the
feedback provided, the responses from blogger, documentations,
etc. Therefore, if the product was invisible or transparent to the
users, how could it act as a medium to help the users to understand
itself? [FIXME: need more words.] Nor can it be engaging or
communicative.

% other direction: user experience
% difficulty of user experience
The community of human-computer interaction quickly embraced the
idea that solely pursuing for interface transparency and performance
is not enough, and a richer model is required to encapsulate the other
perspectives of the interaction, which fall under the umbrella term of
user experience. This may include addressing the human needs beyond
the instrumental; stressing affective and emotional aspects of the
interaction; and dealing with the nature of experience
\citep{ux:hassenzahl}. Many models have been proposed, however the
universal definition for user experience is still not well understood
nor fully clarified. There are several reasons why this is so,
according to \citet{ux:law} for example:
\begin{enumerate}
  \item User experience is associated with a wide range of vague and
    dynamic concepts, including emotional, affective, experiential,
    hedonic, and aesthetic variables. Whether or not to include each
    of these variables is greatly dependent on the interests and
    background of the proposing author.
  \item The unit for measuring user experience is
    \textit{malleable}. Unlike usability where only a single aspect of
    interaction between the user and the product\footnote{Just
      product, not including system nor service.} is assessed with
    well defined criteria, it could be extended to include multiple
    users, environment while the product is being used, and more.
\end{enumerate}

\subsection{Definitions}
Table \ref{tab:definitions} has some sample definitions of user
experience \citep{definition:law}.

\begin{table}[!t]
  \caption{Various definitions of User Experience.}
  \label{tab:definitions}
  \begin{center}
    \begin{tabular}{| p{0.2\columnwidth} || p{0.7\columnwidth} |}
      \hline

      Lauralee Alben: & All the aspects of how people use an
      interactive product: the way it feels in their hands, how well
      they understand how it works, how they feel about it while
      they're using it, how well it serves their purposes, and how
      well it fits into the entire context in which they are using
      it. \\ \hline

      Nielsen Norman Group: & All aspects of the end-user's
      interaction with the company, its services, and its
      product. \\ \hline

      Marc Hassenzahl \& Noam Tractinsky: & A consequence of a user's
      internal state (predispositions, expectations, needs,
      motivation, mood, etc.), the characteristics of the designed
      system (e.g. complexity, purpose, usability, functionality,
      etc.) and the context (or the environment) within which the
      interaction occurs (e.g. organizational/social setting,
      meaningfulness of the activity, voluntariness of use, etc.)
      \\ \hline
    \end{tabular}
  \end{center}
\end{table}

\subsection{Interface Transparency}
[FIXME]
% Talk about various other non interface transparency.

% Talk about Linux and Windows as example for transparency.


\section{Planned Experiment}
\label{sec:experiment}
[FIXME]


\section{Conclusion}
\label{sec:conclusion}
[FIXME]


\bibliographystyle{apalike}
\bibliography{final}


\end{document}
