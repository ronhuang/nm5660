% final.tex
% Final Report for NM5660 Independent Study Module
% Author: Yih-Lun Huang
% Revisions: 16 April 2010

\documentclass{acm_proc_article-sp}

\begin{document}

\title{Investigation of the relationship between transparency and user
  experience}

\numberofauthors{1}
\author{
\alignauthor Yih-Lun Huang\\
\affaddr{HY090191Y}\\
\email{ylhuang@nus.edu.sg}
}

\maketitle
\begin{abstract}
TODO.
\end{abstract}

\category{H.5.2}{Information Interfaces And Presentation}{User
  Interface}[input devices and strategies]

\terms{Design, Algorithms, Measurement}

\keywords{Input device, multi-touch, touchpad, lazy}

\section{Introduction}
When rookie engineers are given the task of designing a new product or
system, most of the time, they tend to squeeze as many features as
possible into whatever they are suppose to design. Interfaces between
the user and the design are scattered all over. This can often be
observed in corporates where a junior software engineer is assigned to
develop an internal tool to facilitate an existing working process
(Figure~\ref{fig:featureful}.) Though the design is usable, the usage
is limited to only a small selected group of users.

\begin{figure}[!t]
\centering
\includegraphics[width=.7\columnwidth]{featureful}
\caption{Internal tool for parsing the content in RAM.}
\label{fig:featureful}
\end{figure}

This is because they think in turn of the technology instead of the
user. They encountered the wonderful of world of engineering and
cannot wait to fully utilize the knowledge they just acquired.




% TODO: describe more on the two groups of people.
Most software engineers who have been through HCI modules are usually
educated to create software with user-friendly, minimalistic and
transparent interactions, which is based on the famous dictum from
Donald Norman \cite{design:norman}. But in fact, there is another
group of people who are claiming that the dictum may be flawed and
incomplete. As suggested from the book by Jay David Bolter and Diane
Gromala \cite{windows:bolter}, these two groups of people, the
structuralists and designers, have their own approaches to formulate
the world of digital designs.

Leading researchers in the HCI fields (structuralists) have been
advocating the importance of transparency of the interaction. They
drew us a future where all the computations are hidden from us. All
the computers will anticipate our needs and react as if there are no
explicit instructions from us. Effectively rendering the computing
devices and technologies invisible.

However, designers on the other hand, claim that transparency is a
myth and has been overly simplified and exaggerated. “The danger of
transparency is that the interface will mask the operation of the
system exactly when the user needs to see and understand what the
system is doing.” They believe that interactions should be an
equivalent between transparency and reflective.

These are both very interesting point of views and prompted people to
think about whether they have been too obsessed with achieving
transparency, the benefits and limitations of transparency, measuring
transparency, and if there is an even grander theory behind this. So
for this independent study module, further exploration of transparent
interactions in HCI will be conducted.

Added to this, the emergence of the user experience is an attempt to
inform design and evaluation has been widespread in HCI. Such that,
researchers are proposing user experience encapsulating the idea of
transparency. Hence, the relationship between transparency and user
experience will also be examined in this ISM.

\section{Related Works}
% TODO: mention Universal usability

\bibliographystyle{abbrv}
\bibliography{final}

\end{document}
